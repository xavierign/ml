\chapter{Conclusion}\label{ch:conclusion}

Since the beginning of the first practice contest about particles with high-energy, our team was really enthusiastic in regards to this competition. In this practice and in the next instance, the one suspended by the controversy, we got outstanding scores. In the real competition about NLP, the third one, we also started heading the official ranking, as we began working since the first days the competition was open. By the second week, we were leading with a performance far above from the full-grade baseline. Hence, we decided that our exploration was enough. We had carried out the application of the tools we have learned from homework's and classes along the course. First, we strictly followed the methodology of not to abuse of the hold-out set, to separate into train, test, validation, cross-validation, etc. Second, we defined a 2-order feature map, representing the bi-grams in a NLP learning problem. Third, we converted the categorical variables into numeric, to be able to feed a wider scope of algorithms. Fourth, we explored a large number of diverse classifiers and obtained the expected results: some ensemble methods out stands from the rest (i.e. Boosting and Bagging). Fifth, we tuned best performer classifiers hyper-parameters by successive executions. Finally, we untied classifiers with similar performances by the parsimony principle to make our submission more robust. We were optimistic about its outcome, but in the last days, and even hours before the closing, some groups improved significantly their performance leaving us away from the top.

Looking back in time and after talking with the leading teams, we learned that there was a little more to be done to get the extra points in the ranking. We realized now that coding by ourselves an ensemble of trained classifiers would have given us the extra classification power we needed to win the competition. We will try this approach next opportunity. 
